\documentclass[10pt,notitlepage]{amsart}

%\usepackage{epsfig}
\usepackage{geometry}
\usepackage{listings}

\usepackage[table]{xcolor}
\definecolor{light-gray}{gray}{0.9}
\geometry{left=1in,top=0.5in,right=1in,bottom=0.5in}


\begin{document}
\lstset{frameround=fttt,frame=trBL,showstringspaces=false}
\noindent {\Large \sc {The RST  (Richard-Spencer-Tenniel) Project }}

\vskip 12pt

\noindent{ {\textsc{Problem Description}}}
\begin{enumerate}
\item Suppose you are given an $m\times n$ matrix $B=(b_{ij})$ such that
\begin{itemize}
  \item $b_{ij}\in\{0,1,2,3, \ldots, R\}$  
  \item $(i_0,j_0), (i_f,j_f)\in \left\{1,2,\ldots,m\right\}\times     \left\{1,2,\ldots, n\right\}$ 
\item  $b_{i_0j_0},b_{i_fj_f}\in\{0,1\}$
\end{itemize}

\item   The matrix $B$ represents a battleground where a robot will start  at $b_{i_0j_0}$ and seek the goal $b_{i_fj_f}$.  Further, an entry of 
\begin{itemize}
  \item  0 designates fog is not present in that position.
  \item  1 designates fog is present in that position.
  \item  2 designates a chasm in that position.
  \item $3, 4, 5, \ldots, R$  designates a later described obstruction.
\end{itemize}
\item The scenario is bound by the following constraints.
\begin{itemize}
  \item Your robot can only move in the $\binom{1}{0}$ and $\binom{0}{1}$ directions.  Hence, any path from $(0,0)$ to $(1,1)$ has length $\geq 2$.
  \item Upon entering an unvisited position, all adjacent cells containing fog will lose their fog designation.
  \item You can only reach your goal position if a (valid) fog-free path  exists from the current position to the goal position.
  \item You cannot enter a cell containing a chasm or type-$3,4,5, \ldots, R$ obstruction.
\end{itemize}
\item The robot will have the following information provided upon initialization
\begin{itemize}
  \item The dimensions $m$ and $n$
  \item The starting position $b_{i_0j_0}$
  \item The goal position $b_{i_fj_f}$  
\end{itemize}
\item The aforementioned obstructions can be interpreted as three (not necessarily distinct) mountain ranges such that
\begin{itemize}
  \item The values of $3, 4, 5, \ldots, R$ determine the boundaries of the minimal encapsulating set\footnote{This is related to the convex hull, but the encapsulating set is not necessarily convex per the forthcoming conventions.} containing those particular instances of $3, 4, 5,\ldots R$.  That is, form the minimal encapsulating set containing 3s, the minimal encapsulating set containing 4s, and so on.  The  minimal encapsulating sets will occupy some non-negative quantity of cells.\footnote{Note there could be zero instances of $3, 4, \ldots, R$.}
  \item The encapsulating sets may have non-empty intersection; form the union of the  encapsulating sets in this case.
  \item In simple terms, we describe the minimal encapsulating set by imagining stretching a rubber band so that it encapsulates all of those squares filled with a 3, 4, etc.  If the rubber band ``holds'' at least half of a given square's area, then it is considered part of the set.  The exception to this is if a particular square contains a chasm.  In this case, it remains a chasm.\footnote{This  clearly has no impact on the solution.}
  
\end{itemize}



\end{enumerate}


\noindent{ {\textsc{Desired Outcome(s)}}}

\begin{itemize}
  \item[] A path (if it exists) from $b_{i_0j_0}$ to $b_{i_fj_f}$.
\end{itemize}

\noindent{ {\textsc{Aesthetics}}}

\begin{itemize}
  \item You should display a graphical representation of the ``map'' and the traversal of the map by the robot as it decides upon its path.
\end{itemize}


\vspace{1cm}
\noindent{ {\textsc{Intended Representation}}}

\vspace{.2cm}

\begin{center}
\begin{tabular}{|c|c|c|c|c|c|c|c|c|}
\hline
  1&1&0&1&1&1&1&1&0\\
  \hline
  0&1&4&1&2&4&1&1&1\\
  \hline  
    1&1&1&2&1&1&4&1&1\\
  \hline    
  1&2&2&4&1&1&1&1&2\\
    \hline
  1&1&1&1&3&1&1&1&2\\
    \hline
  0&1&1&1&1&1&3&1&1\\
    \hline
  0&1&3&1&1&2&1&1&1\\
    \hline
  1&0&2&0&1&2&1&1&1\\              
  \hline
\end{tabular} $\longrightarrow$ \begin{tabular}{|c|c|c|c|c|c|c|c|c|}
\hline
  1&1&0&1&1&1&1&1&0\\
  \hline
  0&1&\cellcolor{blue!25}4&\cellcolor{blue!25}1&\cellcolor{blue!25}2&\cellcolor{blue!25}4&\cellcolor{blue!25}1&1&1\\
  \hline  
    1&1&\cellcolor{blue!25}1&\cellcolor{blue!25}2&\cellcolor{blue!25}1&\cellcolor{blue!25}1&\cellcolor{blue!25}4&1&1\\
  \hline    
  1&\cellcolor{red!25}2&2&\cellcolor{blue!25}4&\cellcolor{blue!25}1&\cellcolor{blue!25}1&1&1&\cellcolor{red!25}2\\
    \hline
  1&1&1&\cellcolor{green!25}1&\cellcolor{green!25}3&\cellcolor{green!25}1&1&1&\cellcolor{red!25}2\\
    \hline
  0&1&\cellcolor{green!25}1&\cellcolor{green!25}1&\cellcolor{green!25}1&\cellcolor{green!25}1&\cellcolor{green!25}3&1&1\\
    \hline
  0&1&\cellcolor{green!25}3&\cellcolor{green!25}1&\cellcolor{green!25}1&\cellcolor{red!25}2&1&1&1\\
    \hline
  1&0&\cellcolor{red!25}2&0&1&\cellcolor{red!25}2&1&1&1\\              
  \hline
\end{tabular} $\longrightarrow$ \begin{tabular}{|c|c|c|c|c|c|c|c|c|}
\hline
  1&1&0&1&1&1&1&1&0\\
  \hline
  0&1&\cellcolor{blue!25}4&\cellcolor{blue!25}4&\cellcolor{blue!25}4&\cellcolor{blue!25}4&\cellcolor{blue!25}4&1&1\\
  \hline  
    1&1&\cellcolor{blue!25}4&\cellcolor{blue!25}4&\cellcolor{blue!25}4&\cellcolor{blue!25}4&\cellcolor{blue!25}4&1&1\\
  \hline    
  1&\cellcolor{red!25}2&2&\cellcolor{blue!25}4&\cellcolor{blue!25}4&\cellcolor{blue!25}4&1&1&\cellcolor{red!25}2\\
    \hline
  1&1&1&\cellcolor{green!25}3&\cellcolor{green!25}3&\cellcolor{green!25}3&1&1&\cellcolor{red!25}2\\
    \hline
  0&1&\cellcolor{green!25}3&\cellcolor{green!25}3&\cellcolor{green!25}3&\cellcolor{green!25}3&\cellcolor{green!25}3&1&1\\
    \hline
  0&1&\cellcolor{green!25}3&\cellcolor{green!25}3&\cellcolor{green!25}3&\cellcolor{red!25}2&1&1&1\\
    \hline
  1&0&\cellcolor{red!25}2&0&1&\cellcolor{red!25}2&1&1&1\\              
  \hline
\end{tabular} 
\end{center}


\end{document}
